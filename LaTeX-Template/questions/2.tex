\subsectionaddtolist{معماری استاندارد TCP/IP و مقایسه با مدل OSI}

استاندارد \lr{TCP/IP (Transmission Control Protocol/Internet Protocol)} مجموعه پروتکل‌هایی است که اساس عملکرد اینترنت را تشکیل می‌دهند. این استاندارد که در ابتدا توسط وزارت دفاع ایالات متحده توسعه یافت، به دلیل قابلیت اطمینان، انعطاف‌پذیری و مقیاس‌پذیری بالا، به استاندارد اصلی ارتباطات شبکه‌ای تبدیل شد.

\subsubsection*{معماری استاندارد TCP/IP و وظایف هر لایه}

مدل TCP/IP یک معماری لایه‌ای است که به وظایف ارتباطات شبکه را به بخش‌های کوچک‌تر و قابل مدیریت تقسیم می‌کند. برخلاف مدل هفت لایه‌ای OSI ، مدل TCP/IP به طور سنتی دارای چهار یا پنج لایه است که در ادامه به تشریح آنها می‌پردازیم:

\begin{enumerate}
	\item \textbf{لایه کاربرد (\lr{Application Layer}):}
	این لایه بالاترین لایه در مدل TCP/IP است و مسئول ارائه خدمات شبکه به برنامه‌های کاربردی است. پروتکل‌های این لایه با کاربران نهایی و نرم‌افزارهای آنها در تعامل هستند. این لایه وظایف لایه‌های Application ، Presentation و Session در مدل OSI را ترکیب می‌کند.

	\begin{itemize}
		\item \textbf{وظایف اصلی:}
		\begin{itemize}
			\item پشتیبانی از برنامه‌های کاربردی شبکه مانند مرورگرهای وب، برنامه‌های ایمیل، نرم‌افزارهای انتقال فایل و غیره.
			
			\item مدیریت نمایش داده‌ها (مانند فشرده‌سازی، رمزنگاری و قالب‌بندی) به گونه‌ای که برای برنامه‌های کاربردی قابل فهم باشد.
			
			\item برقراری، مدیریت و پایان دادن به جلسات ارتباطی بین برنامه‌ها.
		\end{itemize}
		
		\item \textbf{پروتکل‌های نمونه:} \lr{HTTP, HTTPS, FTP, SMTP, DNS, SSH, Telnet}.
	\end{itemize}
	
	\item \textbf{لایه انتقال (\lr{Transport Layer}):}
	این لایه مسئول برقراری ارتباط سرتاسری (End-to-End) بین فرایندها در سیستم‌های مبدأ و مقصد است. وظیفه اصلی آن اطمینان از تحویل مطمئن و منظم داده‌ها یا انتقال سریع آنها بدون تضمین تحویل است.
	\begin{itemize}
		\item \textbf{وظایف اصلی:}
		\begin{itemize}
			\item \textbf{تقسیم‌بندی داده‌ها:} تقسیم داده‌ها از لایه کاربرد به قطعات کوچک‌تر به نام سگمنت و بازسازی آنها در مقصد.
			
			\item \textbf{مالتی‌پلکسینگ/دیمالتی‌پلکسینگ:} امکان اجرای چندین برنامه کاربردی به طور همزمان و ارسال و دریافت داده‌ها از طریق یک اتصال شبکه مشترک.
			
			\item \textbf{کنترل جریان:} جلوگیری از سرریز شدن گیرنده با ارسال داده‌ها با سرعتی که گیرنده قادر به پردازش آن باشد.
			
			\item \textbf{کنترل خطا:} اطمینان از رسیدن داده‌ها به مقصد بدون خطا و به ترتیب صحیح.
		\end{itemize}
		
		\item \textbf{پروتکل‌های نمونه:}
		\begin{itemize}
			\item \textbf{TCP (\lr{Transmission Control Protocol}):} یک پروتکل اتصال‌‌گرا‌ و قابل اطمینان است. قبل از ارسال داده‌ها، یک اتصال مجازی بین مبدأ و مقصد برقرار می‌کند و تحویل صحیح و به ترتیب داده‌ها را تضمین می‌کند. برای کاربردهایی که به دقت داده‌ها اهمیت می‌دهند (مانند مرور وب، ایمیل، انتقال فایل) مناسب است.
			
			\item \textbf{UDP (\lr{User Datagram Protocol}):} یک پروتکل بدون اتصال و غیرقابل اطمینان است. داده‌ها را بدون برقراری اتصال قبلی ارسال می‌کند و هیچ تضمینی برای تحویل، ترتیب یا عدم تکرار بسته ارائه نمی‌دهد. برای کاربردهایی که سرعت مهم‌تر از دقت است ( مانند پخش زنده ویدئو، بازی‌های آنلاین، VoIP ) استفاده می‌شود.
		\end{itemize}
	\end{itemize}
	
	\item \textbf{لایه شبکه (\lr{Network Layer}):}
	این لایه مسئول آدرس‌دهی منطقی (\lr{IP Address}) و مسیریابی (\lr{Routing}) بسته‌های داده از مبدأ به مقصد، حتی اگر در شبکه‌های مختلفی قرار داشته باشند، است.
	\begin{itemize}
		\item \textbf{وظایف اصلی:}
		\begin{itemize}
			\item \textbf{آدرس‌دهی منطقی:} تخصیص آدرس‌های IP به دستگاه‌ها برای شناسایی منحصر به فرد آنها در شبکه.
			\item \textbf{مسیریابی:} تعیین بهترین مسیر برای ارسال بسته‌های داده از طریق شبکه‌های مختلف (با استفاده از روترها).
			\item \textbf{قطعه‌بندی (\lr{Fragmentation}):} در صورت نیاز، تقسیم بسته‌های بزرگتر به قطعات کوچکتر برای سازگاری با حداکثر واحد انتقال (\lr{MTU}) در لینک‌های مختلف شبکه.
		\end{itemize}
		\item \textbf{پروتکل‌های نمونه:} IP (\lr{Internet Protocol}) که مهمترین پروتکل این لایه است. ARP (\lr{Address Resolution Protocol}), ICMP (\lr{Internet Control Message Protocol}), IGMP (\lr{Internet Group Management Protocol}) نیز در این لایه فعالیت می‌کنند.
	\end{itemize}
	
	\item \textbf{لایه دسترسی به شبکه (\lr{Data Link}):}
	این لایه پایین‌ترین لایه در مدل \lr{TCP/IP} است و مسئول جزئیات فیزیکی نحوه اتصال دستگاه به شبکه و نحوه انتقال داده‌ها از طریق واسط فیزیکی است. این لایه وظایف لایه‌های \lr{Data Link} و \lr{Physical} در مدل OSI را ترکیب می‌کند.
	\begin{itemize}
		\item \textbf{وظایف اصلی:}
		\begin{itemize}
			\item \textbf{مدیریت دسترسی به رسانه (\lr{Media Access Control - MAC}):} کنترل نحوه دسترسی دستگاه‌ها به رسانه فیزیکی (کابل، فیبر، امواج رادیویی).
			\item \textbf{آدرس‌دهی فیزیکی (\lr{MAC Address}):} استفاده از آدرس‌های فیزیکی (\lr{MAC}) برای شناسایی دستگاه‌ها در یک شبکه محلی.
			\item \textbf{کنترل خطا در سطح فریم:} تشخیص و گاهی اوقات تصحیح خطاهای انتقال داده در یک لینک فیزیکی.
			\item \textbf{تبدیل بیت‌ها به سیگنال‌های فیزیکی:} تبدیل بیت‌های داده به سیگنال‌های الکتریکی، نوری یا رادیویی برای انتقال و برعکس.
		\end{itemize}
		\item \textbf{پروتکل‌های نمونه:} \lr{Ethernet}, \lr{Wi-Fi (802.11)}, \lr{PPP (Point-to-Point Protocol)}, \lr{ATM}, \lr{Frame Relay}.
	\end{itemize}
	
	\item \textbf{لایه فیزیکی (\lr{Physical Layer}):}
	این لایه پایین‌ترین لایه در مدل TCP/IP است و مسئول جزئیات فیزیکی نحوه اتصال دستگاه به شبکه و نحوه انتقال بیت‌ها از طریق واسط فیزیکی است. این لایه مستقیماً با رسانه انتقال (مانند کابل، فیبر نوری، یا امواج رادیویی) در تعامل است.
	\begin{itemize}
		\item \textbf{وظایف اصلی:}
		\begin{itemize}
			\item \textbf{تبدیل بیت‌ها به سیگنال‌های فیزیکی:} تبدیل بیت‌های داده به سیگنال‌های الکتریکی، نوری یا رادیویی برای انتقال و برعکس.
			\item \textbf{مشخصات فیزیکی:} تعریف مشخصات فیزیکی کابل‌ها، کانکتورها، ولتاژها و نرخ بیت‌ها.
			\item \textbf{توپولوژی فیزیکی:} تعیین نحوه اتصال فیزیکی دستگاه‌ها در شبکه.
		\end{itemize}
		\item \textbf{پروتکل‌ها و استانداردهای نمونه:} \lr{Ethernet} (مشخصات فیزیکی مانند \lr{10BASE-T}, \lr{100BASE-TX}, \lr{1000BASE-T}), استانداردهای \lr{USB}, \lr{Bluetooth}, \lr{RS-232}.
	\end{itemize}
\end{enumerate}

\subsubsection*{مقایسه معماری TCP/IP با معماری OSI}

هر دو مدل TCP/IP و OSI مدل‌های لایه‌ای هستند که برای توضیح عملکرد ارتباطات شبکه طراحی شده‌اند، اما تفاوت‌های کلیدی بین آنها وجود دارد:

\begin{itemize}
	\item \textbf{تعداد لایه‌ها:}
	\begin{itemize}
		\item \textbf{مدل OSI :} دارای ۷ لایه (فیزیکی، پیوند داده، شبکه، انتقال، جلسه، نمایش، کاربرد).
		\item \textbf{مدل TCP/IP :} به طور سنتی دارای ۵ لایه است (دسترسی به شبکه، اینترنت، انتقال، کاربرد، فیزیکی).
	\end{itemize}
	
	\item \textbf{فلسفه طراحی:}
	\begin{itemize}
		\item \textbf{مدل OSI :} یک مدل مفهومی و نظری است که به عنوان یک استاندارد مرجع برای نحوه عملکرد شبکه‌ها توسعه یافت. هدف آن ارائه یک چارچوب جامع و مستقل از پروتکل بود. ابتدا مدل طراحی شد و سپس پروتکل‌ها برای آن نوشته شدند.
		\item \textbf{مدل TCP/IP :} یک مدل عملیاتی است که بر اساس پروتکل‌های واقعی توسعه یافت. ابتدا پروتکل‌ها (برای \lr{ARPANET}) ایجاد شدند و سپس مدل برای توصیف آنها طراحی شد. این مدل بیشتر بر روی کاربردی بودن و پیاده‌سازی تمرکز دارد.
	\end{itemize}
	
	\item \textbf{ترکیب لایه‌ها:}
	\begin{itemize}
		\item \textbf{مدل OSI :} وظایف به وضوح بین ۷ لایه تفکیک شده‌اند.
		\item \textbf{مدل TCP/IP :} لایه‌هایی از OSI را با هم ترکیب می‌کند:
		\begin{itemize}
			\item لایه‌های فیزیکی و پیوند داده OSI در لایه دسترسی به شبکه TCP/IP ادغام شده‌اند.
			\item لایه‌های جلسه، نمایش و کاربرد OSI در لایه کاربرد TCP/IP ادغام شده‌اند.
		\end{itemize}
	\end{itemize}
	
	\item \textbf{ماهیت اتصال:}
	\begin{itemize}
		\item \textbf{مدل OSI :} لایه انتقال آن می‌تواند هم اتصال‌گرا و هم بدون اتصال باشد.
		\item \textbf{مدل TCP/IP :} لایه انتقال آن شامل هر دو پروتکل TCP (اتصال‌گرا) و UDP (بدون اتصال) است، و لایه اینترنت آن (IP) ذاتاً بدون اتصال است.
	\end{itemize}
	
	\item \textbf{استفاده در عمل:}
	\begin{itemize}
		\item \textbf{مدل OSI :} بیشتر برای مقاصد آموزشی و تئوریک، و برای درک نحوه عملکرد شبکه‌ها به صورت جامع استفاده می‌شود.
		\item \textbf{مدل TCP/IP :} به دلیل ارتباط نزدیک با پروتکل‌های واقعی اینترنت، به طور گسترده‌ای در پیاده‌سازی‌های عملی شبکه و به ویژه اینترنت مورد استفاده قرار می‌گیرد.
	\end{itemize}
	
	\item \textbf{پروتکل‌ها و خدمات:}
	\begin{itemize}
		\item \textbf{مدل OSI :} تمایز واضحی بین سرویس‌ها، رابط‌ها و پروتکل‌ها وجود دارد.
		\item \textbf{مدل TCP/IP :} این تمایزها کمتر مشخص هستند و پروتکل‌ها جزو لاینفک هر لایه محسوب می‌شوند.
	\end{itemize}
\end{itemize}

\textbf{نتیجه‌گیری:}
هر دو مدل TCP/IP و OSI درک جامعی از فرآیندهای ارتباطی در شبکه‌ها ارائه می‌دهند. در حالی که مدل OSI یک چارچوب مفهومی دقیق‌تر است که به صورت تئوری به وظایف ارتباطات شبکه می‌پردازد، مدل TCP/IP یک مدل عملیاتی و کاربردی است که مستقیماً به پروتکل‌های مورد استفاده در اینترنت اشاره دارد. امروزه، اینترنت و بیشتر شبکه‌های عملیاتی بر اساس معماری TCP/IP کار می‌کنند.