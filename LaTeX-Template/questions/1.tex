\subsectionaddtolist{مقایسه کابل‌های کواکسیال، زوج سیم به هم تابیده و فیبر نوری}


\subsubsection*{سرعت انتقال داده}

\begin{itemize}
	\item \textbf{کابل کواکسیال:} سرعت انتقال داده در کابل‌های کواکسیال، بسته به نوع و کیفیت کابل، از حدود ۱۰ مگابیت بر ثانیه تا چند صد مگابیت بر ثانیه متغیر است. این کابل‌ها در مقایسه با زوج سیم به هم تابیده سرعت بالاتری ارائه می‌دهند، اما به مراتب کندتر از فیبر نوری هستند.
	
	\item \textbf{کابل زوج سیم به هم تابیده:} سرعت انتقال داده در این کابل‌ها نیز بسته به رده (Category) کابل متفاوت است. به عنوان مثال، \lr{Cat5e} تا ۱ گیگابیت بر ثانیه، \lr{Cat6} تا ۱۰ گیگابیت بر ثانیه (در فواصل کوتاه) و \lr{Cat6a}، \lr{Cat7} و \lr{Cat8} تا ۱۰ گیگابیت بر ثانیه و بالاتر را پشتیبانی می‌کنند. این کابل‌ها برای شبکه‌های محلی (LAN) بسیار رایج هستند و سرعت‌های مناسبی را برای بسیاری از کاربردها فراهم می‌کنند.
	
	\item \textbf{فیبر نوری:} فیبر نوری از نظر سرعت انتقال داده، بهترین عملکرد را دارد. سرعت انتقال در فیبر نوری به دلیل استفاده از نور به جای سیگنال‌های الکتریکی، می‌تواند به گیگابیت‌ها و حتی ترابیت‌ها در ثانیه برسد. این کابل‌ها برای فواصل طولانی و پهنای باند بسیار بالا ایده‌آل هستند و محدودیتی از نظر سرعت در کاربردهای معمول ندارند.
\end{itemize}

\subsubsection*{احتمال ایجاد خطا}

\begin{itemize}
	\item \textbf{کابل کواکسیال:} کابل‌های کواکسیال به دلیل ساختار محافظتی (شیلد) خود، در برابر نویزهای الکترومغناطیسی و تداخل خارجی مقاوم‌تر از زوج سیم به هم تابیده بدون شیلد هستند. با این حال، همچنان مستعد تداخل و کاهش کیفیت سیگنال در فواصل طولانی یا در محیط‌های با نویز بالا هستند.
	
	\item \textbf{کابل زوج سیم به هم تابیده:} در کابل‌های زوج سیم به هم تابیده، پیچش سیم‌ها به کاهش تداخل الکترومغناطیسی کمک می‌کند. 
	با این حال، این کابل‌ها (به ویژه انواع بدون شیلد) در برابر تداخل خارجی آسیب‌پذیرتر هستند. انواع شیلددار مقاومت بیشتری در برابر نویز خارجی دارند، اما گران‌تر و نصب آنها دشوارتر است.
	
	\item \textbf{فیبر نوری:} فیبر نوری از سیگنال‌های نوری استفاده می‌کند و به همین دلیل کاملاً در برابر تداخلات الکترومغناطیسی و رادیویی مقاوم است. این ویژگی باعث می‌شود که احتمال ایجاد خطا در فیبر نوری بسیار پایین باشد، حتی در محیط‌های با نویز بالا. این مزیت، فیبر نوری را برای محیط‌های صنعتی، نظامی و پزشکی بسیار مناسب می‌سازد.
\end{itemize}

\subsubsection*{میزان کاهش انرژی سیگنال}

\begin{itemize}
	\item \textbf{کابل کواکسیال:} کابل کواکسیال دارای تضعیف سیگنال کمتری نسبت به زوج سیم به هم تابیده در فواصل مشابه است. با این حال، با افزایش طول کابل و فرکانس سیگنال، تضعیف نیز افزایش می‌یابد و برای فواصل طولانی نیاز به تقویت‌کننده وجود خواهد داشت.

	\item \textbf{کابل زوج سیم به هم تابیده:} تضعیف سیگنال در کابل‌های زوج سیم به هم تابیده نسبت به کواکسیال بیشتر است، به خصوص در فرکانس‌های بالاتر و فواصل طولانی‌تر. به همین دلیل، حداکثر طول مجاز برای این کابل‌ها (معمولاً ۱۰۰ متر برای شبکه‌های اترنت) محدود است و پس از آن نیاز به تجهیزات (مانند سوئیچ یا روتر) برای بازسازی سیگنال است.

	\item \textbf{فیبر نوری:} فیبر نوری کمترین میزان تضعیف سیگنال را در بین این سه نوع کابل دارد. تضعیف در فیبر نوری بسیار ناچیز است و امکان انتقال داده در فواصل بسیار طولانی (کیلومترها) را بدون نیاز به تقویت‌کننده فراهم می‌کند. این ویژگی، فیبر نوری را برای شبکه‌های گسترده (WAN) و اتصال بین مراکز داده بسیار مناسب می‌سازد.
\end{itemize}

\subsubsection*{شرایط توجیه‌پذیری و مقرون به صرفه بودن استفاده}

\begin{itemize}
	\item \textbf{کابل کواکسیال:}
	\begin{itemize}
		\item \textbf{شرایط توجیه‌پذیری:} در حال حاضر، استفاده از کابل کواکسیال برای شبکه‌های کامپیوتری (به جز در موارد خاص و قدیمی) بسیار محدود شده است. با این حال، این کابل همچنان در سیستم‌های تلویزیون کابلی و برخی سیستم‌های نظارت تصویری (آنالوگ) و یا در اتصالات کوتاه برای انتقال فرکانس‌های رادیویی توجیه‌پذیر و مقرون به صرفه است.
		
		\item \textbf{مقرون به صرفه بودن:} از نظر هزینه اولیه، کابل کواکسیال ارزان‌تر از فیبر نوری و کمی گران‌تر از زوج سیم به هم تابیده است. هزینه نصب آن نیز نسبتاً پایین است.
	\end{itemize}
	
	\item \textbf{کابل زوج سیم به هم تابیده:}
	\begin{itemize}
		\item \textbf{شرایط توجیه‌پذیری:} این کابل‌ها در حال حاضر، رایج‌ترین و مقرون به صرفه‌ترین گزینه برای شبکه‌های محلی (LAN) در محیط‌های اداری، خانگی و تجاری کوچک و متوسط هستند. برای فواصل کوتاه (تا ۱۰۰ متر) و نیاز به سرعت‌های گیگابیتی، \lr{Cat5e} و \lr{Cat6} گزینه‌های بسیار مناسبی هستند. برای نیازهای بالاتر و فواصل کمی طولانی‌تر، \lr{Cat6a}، \lr{Cat7} و \lr{Cat8} نیز قابل استفاده‌اند.
		
		\item \textbf{مقرون به صرفه بودن:} از نظر هزینه اولیه کابل و تجهیزات شبکه (مانند کارت شبکه و سوئیچ)، زوج سیم به هم تابیده بسیار مقرون به صرفه است. نصب آن نیز نسبتاً آسان است و به ابزار و مهارت‌های خاصی نیاز ندارد.
	\end{itemize}
	
	\item \textbf{فیبر نوری:}
	\begin{itemize}
		\item \textbf{شرایط توجیه‌پذیری:} استفاده از فیبر نوری در مواردی که نیاز به سرعت‌های بسیار بالا، انتقال داده در فواصل طولانی (بیش از ۱۰۰ متر)، امنیت بالا در برابر استراق سمع، یا مقاومت کامل در برابر تداخلات الکترومغناطیسی وجود دارد، کاملاً توجیه‌پذیر است. این شرایط شامل شبکه‌های ستون فقرات در سازمان‌ها، ارتباطات بین ساختمان‌ها، شبکه‌های شهری، شبکه‌های گسترده، مراکز داده و کاربردهای صنعتی و نظامی می‌شود.
		
		\item \textbf{مقرون به صرفه بودن:} هزینه اولیه کابل و تجهیزات فیبر نوری (مانند کارت شبکه فیبر نوری، سوئیچ‌های فیبر نوری، و تجهیزات جوش فیبر) به مراتب گران‌تر از کابل‌های مسی است. همچنین، نصب و نگهداری فیبر نوری نیازمند تخصص و ابزارهای خاص است که هزینه کلی را افزایش می‌دهد. با این حال، با توجه به پهنای باند و فاصله انتقال بی‌نظیر، در بلندمدت و برای کاربردهای خاص، فیبر نوری می‌تواند از نظر اقتصادی بسیار مقرون به صرفه باشد، زیرا نیاز به تقویت‌کننده‌های کمتری دارد و طول عمر بالاتری دارد.
	\end{itemize}
\end{itemize}